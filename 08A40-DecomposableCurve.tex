\documentclass[12pt]{article}
\usepackage{pmmeta}
\pmcanonicalname{DecomposableCurve}
\pmcreated{2013-03-22 19:19:38}
\pmmodified{2013-03-22 19:19:38}
\pmowner{pahio}{2872}
\pmmodifier{pahio}{2872}
\pmtitle{decomposable curve}
\pmrecord{7}{42268}
\pmprivacy{1}
\pmauthor{pahio}{2872}
\pmtype{Definition}
\pmcomment{trigger rebuild}
\pmclassification{msc}{08A40}
\pmclassification{msc}{26A09}
\pmrelated{Hyperbola2}
\pmdefines{decomposable}
\pmdefines{decomposable surface}

\endmetadata

% this is the default PlanetMath preamble.  as your knowledge
% of TeX increases, you will probably want to edit this, but
% it should be fine as is for beginners.

% almost certainly you want these
\usepackage{amssymb}
\usepackage{amsmath}
\usepackage{amsfonts}

% used for TeXing text within eps files
%\usepackage{psfrag}
% need this for including graphics (\includegraphics)
%\usepackage{graphicx}
% for neatly defining theorems and propositions
 \usepackage{amsthm}
% making logically defined graphics
%%%\usepackage{xypic}

% there are many more packages, add them here as you need them

% define commands here

\theoremstyle{definition}
\newtheorem*{thmplain}{Theorem}

\begin{document}
An algebraic curve
$$f(x,\,y) \;=\; 0$$
is \emph{decomposable}, if the polynomial \,$f(x,\,y)$\, is \PMlinkescapetext{reducible} in\, $\mathbb{R}[x,\,y]$; that is, if there are polynomials\, $g(x,\,y)$\, and \,$h(x,\,y)$\, with positive degree in\, $\mathbb{R}[x,\,y]$\, such that 
$$f(x,\,y) \;=\; g(x,\,y)\,h(x,\,y).$$

\textbf{Example.}\, The quadratic curve
\begin{align}
\frac{x^2}{a^2}\!-\!\frac{y^2}{b^2} \;=\; 0
\end{align}
is decomposable, since the equation may be written
$$\left(\frac{x}{a}\!+\!\frac{y}{b}\right)\!\left(\frac{x}{a}\!-\!\frac{y}{b}\right) \;=\; 0$$
or equivalently
$$\frac{x}{a}\!+\!\frac{y}{b} \;=\; 0 \quad \lor \quad \frac{x}{a}\!-\!\frac{y}{b} \;=\; 0.$$
Thus the curve (1) consists of two intersecting lines.\\

Analogically, one can say that an algebraic surface 
$$g(x,\,y,\,z) \;=\; 0$$
is \emph{decomposable}, e.g.\, $(x\!+\!y\!+\!z)^2\!-\!1 = 0$\, which consists of two parallel planes.

%%%%%
%%%%%
\end{document}
