\documentclass[12pt]{article}
\usepackage{pmmeta}
\pmcanonicalname{BinaryOperation}
\pmcreated{2013-03-22 13:08:12}
\pmmodified{2013-03-22 13:08:12}
\pmowner{mclase}{549}
\pmmodifier{mclase}{549}
\pmtitle{binary operation}
\pmrecord{7}{33574}
\pmprivacy{1}
\pmauthor{mclase}{549}
\pmtype{Definition}
\pmcomment{trigger rebuild}
\pmclassification{msc}{08A99}
\pmsynonym{internal composition}{BinaryOperation}
\pmrelated{Arity}
\pmrelated{Operation}

\endmetadata

% this is the default PlanetMath preamble.  as your knowledge
% of TeX increases, you will probably want to edit this, but
% it should be fine as is for beginners.

% almost certainly you want these
\usepackage{amssymb}
\usepackage{amsmath}
\usepackage{amsfonts}

% used for TeXing text within eps files
%\usepackage{psfrag}
% need this for including graphics (\includegraphics)
%\usepackage{graphicx}
% for neatly defining theorems and propositions
%\usepackage{amsthm}
% making logically defined graphics
%%%\usepackage{xypic}

% there are many more packages, add them here as you need them

% define commands here
\begin{document}
A \emph{binary operation} on a set $X$ is a function from the Cartesian product $X \times X$ to $X$.  A binary operation is sometimes called \emph{internal composition}.

Rather than using function notation, it is usual to write binary operations with an operation symbol between elements, or even with no operation at all, it being understood that juxtaposed elements are to be combined using an operation that should be clear from the context.

Thus, addition of real numbers is the operation
$$(x, y) \mapsto x + y,$$
and multiplication in a groupoid is the operation
$$(x, y) \mapsto xy.$$
%%%%%
%%%%%
\end{document}
