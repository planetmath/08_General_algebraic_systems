\documentclass[12pt]{article}
\usepackage{pmmeta}
\pmcanonicalname{TopicEntryOnTheAlgebraicFoundationsOfMathematics}
\pmcreated{2013-03-22 18:14:02}
\pmmodified{2013-03-22 18:14:02}
\pmowner{bci1}{20947}
\pmmodifier{bci1}{20947}
\pmtitle{topic entry on the algebraic foundations of mathematics}
\pmrecord{35}{40822}
\pmprivacy{1}
\pmauthor{bci1}{20947}
\pmtype{Topic}
\pmcomment{trigger rebuild}
\pmclassification{msc}{08A99}
\pmclassification{msc}{08A70}
\pmclassification{msc}{18E05}
\pmclassification{msc}{18-00}
\pmclassification{msc}{03-00}
\pmclassification{msc}{08A05}
\pmsynonym{Algebraic Foundations of Mathematics}{TopicEntryOnTheAlgebraicFoundationsOfMathematics}
%\pmkeywords{topic entry on the Algebraic Foundations of Mathematics}
\pmrelated{Algebras}
\pmrelated{Graph}
\pmrelated{Hypergraph}
\pmrelated{TopicEntryOnAlgebra}
\pmrelated{IndexOfCategoryTheory}
\pmrelated{NonAbelianStructures}
\pmrelated{JordanBanachAndJordanLieAlgebras}
\pmrelated{AbelianCategory}
\pmrelated{AxiomsForAnAbelianCategory}
\pmrelated{GeneralizedVanKampenTheoremsHigherDimensional}
\pmrelated{AxiomaticTheoryOfSupercategories}
\pmrelated{Categ}
\pmdefines{universal algebra}
\pmdefines{algebraic structure}
\pmdefines{logic algebra}
\pmdefines{co-algebra}
\pmdefines{gebra}
\pmdefines{K-algebra}
\pmdefines{quantum algebra}
\pmdefines{lattice algebra}

\endmetadata

% this is the default PlanetMath preamble.  as your 

\usepackage{amssymb}
\usepackage{amsmath}
\usepackage{amsfonts}

% used for TeXing text within eps files
%\usepackage{psfrag}
% need this for including graphics (\includegraphics)
%\usepackage{graphicx}
% for neatly defining theorems and propositions
%\usepackage{amsthm}
% making logically defined graphics
%%%\usepackage{xypic}

% there are many more packages, add them here as you need them

% define commands here
% this is the default PlanetMath preamble.  as your knowledge
% of TeX increases, you will probably want to edit this, but
% it should be fine as is for beginners.

% almost certainly you want these
\usepackage{amssymb}
\usepackage{amsmath}
\usepackage{amsfonts}

% used for TeXing text within eps files
%\usepackage{psfrag}
% need this for including graphics (\includegraphics)
%\usepackage{graphicx}
% for neatly defining theorems and propositions
%\usepackage{amsthm}
% making logically defined graphics
%%%\usepackage{xypic}

% there are many more packages, add them here as you need them

% define commands here
\usepackage{amssymb,amscd}
\usepackage{amsmath}
\usepackage{amsfonts}
\usepackage{mathrsfs}

% used for TeXing text within eps files
%\usepackage{psfrag}
% need this for including graphics (\includegraphics)
%\usepackage{graphicx}
% for neatly defining theorems and propositions
\usepackage{amsthm}
% making logically defined graphics
%%\usepackage{xypic}
\usepackage{pst-plot}

% define commands here
\newcommand*{\abs}[1]{\left\lvert #1\right\rvert}
\newtheorem{prop}{Proposition}
\newtheorem{thm}{Theorem}
\newtheorem{ex}{Example}
\newcommand{\real}{\mathbb{R}}
\newcommand{\pdiff}[2]{\frac{\partial #1}{\partial #2}}
\newcommand{\mpdiff}[3]{\frac{\partial^#1 #2}{\partial #3^#1}}

\begin{document}
This is a contributed topic on the algebraic foundations of mathematics. This topic of algebraic foundations in mathematics will cover a wide range of concepts and areas of mathematics, ranging from universal algebras, algebraic topology to algebraic geometry, number theory and logic algebras. 

\textbf{a.}  \emph{Universal (or general) algebra} : is defined as \emph{the (meta) mathematical study of general theories of algebraic structures} rather than the study of specific cases, or models of algebraic structures. 

\textbf{b.}  Various, specifically selected algebraic structures, such as :

\begin{enumerate}
\item Boolean algebra 

\item Logic lattice algebras or many-valued (MV)  logic algebras 

\item Quantum logic algebras

\item Quantum operator algebras ( such as :   involution, *-algebras, or $*$-algebras, von Neumann algebras,
JB- and JL- algebras, Poisson and  $C^*$ - or C*- algebras, 

\item Algebra over a set 

\item Sigma-algebra and T-algebras of monads

\item K-algebras

\item Group algebras

\item Graphs generated by free groups

\item Groupoid algebras and Groupoid $C^*$-convolution algebras

\item Hypergraphs generated by free groupoids

\item Double algebras

\item Index of algebras

\item Categorical algebra

\item F-algebra/coalgebra in category theory

\item Category of categories as a foundation for mathematics: \PMlinkname{Functor Categories}{FunctorCategories} and \PMlinkname{2-category}{2Category}

\item \PMlinkname{Index of category theory}{IndexOfCategoryTheory}

\item super-categories and topological `supercategories'

\item Higher dimensional algebras (HDA) --such as:  algebroids, double algebroids, categorical algebroids, double groupoid convolution algebroids, groupoid $C^*$ -convolution algebroids, etc.,  and Supercategorical algebras (SA) as concrete interpretations of the theory of elementary abstract supercategories (ETAS)
\item Index of supercategories

\item \PMlinkname{Index of categories}{IndexOfCategories}

\item Index of HDA

\end{enumerate}

\textbf{Remark}  The last items of HDA  and SA are more precisely understood in the context of, or as generalizations/ extensions of, universal algebras. 


\begin{thebibliography} {9}

\bibitem{AS}
Alfsen, E.M. and F. W. Schultz: \emph{Geometry of State Spaces of Operator Algebras}, Birkh\"auser, Boston--Basel--Berlin (2003).

\bibitem{AMF56}
Atyiah, M.F. 1956. On the Krull-Schmidt theorem with applications to sheaves.
\emph{Bull. Soc. Math. France}, \textbf{84}: 307--317.

\bibitem{AMF56}
Auslander, M. 1965. Coherent Functors. \emph{Proc. Conf. Cat. Algebra, La Jolla},
189--231.
  
\bibitem{AS-BC2k}
Awodey, S. \& Butz, C., 2000, Topological Completeness for Higher Order Logic., Journal of Symbolic Logic, 65, 3, 1168--1182. 

\bibitem{AS-RER2k2}
Awodey, S. \& Reck, E. R., 2002, Completeness and Categoricity I. 
Nineteen-Century Axiomatics to Twentieth-Century Metalogic., History and Philosophy of Logic, 23, 1, 1--30.
  
\bibitem{AS-RER2k2}
Awodey, S. \& Reck, E. R., 2002, ``Completeness and Categoricity II. Twentieth-Century Metalogic to Twenty-first-Century Semantics'', History and Philosophy of Logic, 23, 2, 77--94.  

\bibitem{AS96}
``Structure in Mathematics and Logic: A Categorical Perspective'', Philosophia Mathematica, 3, 209--237. 

\bibitem{AS2k4}
Awodey, S., 2004, ``An Answer to Hellman's Question: Does Category Theory Provide a Framework for Mathematical Structuralism'', Philosophia Mathematica, 12, 54--64. 

\bibitem{AS2k6}
Awodey, S., 2006, Category Theory, Oxford: Clarendon Press. 

\bibitem{BAJ-DJ98a}
Baez, J. \& Dolan, J., 1998a, ``Higher-Dimensional Algebra III. n-Categories and the Algebra of Opetopes'', Advances in Mathematics, 135, 145--206.  


\bibitem{BAJ-DJ2k1}
Baez, J. \& Dolan, J., 2001, ``From Finite Sets to Feynman Diagrams'', Mathematics Unlimited -- 2001 and Beyond, Berlin: Springer, 29--50.  

\bibitem{BAJ-DJ97}
Baez, J., 1997, ``An Introduction to n-Categories'', Category Theory and Computer Science, Lecture Notes in Computer Science, 1290, Berlin: Springer-Verlag, 1--33. 

\bibitem{ICB3}
Baianu, I.C.: 1970, Organismic Supercategories: II. On Multistable Systems. \emph{Bulletin of Mathematical Biophysics}, \textbf{32}: 539-561.
 
\bibitem{ICB4}
Baianu, I.C.: 1971b, Categories, Functors and Quantum Algebraic
Computations, in P. Suppes (ed.), \emph{Proceed. Fourth Intl. Congress Logic-Mathematics-Philosophy of Science}, September 1--4, 1971, Bucharest.

\bibitem{ICBs5}
Baianu, I.C. and D. Scripcariu: 1973, On Adjoint Dynamical Systems. \emph{Bulletin of Mathematical Biophysics}, \textbf{35}(4), 475--486.

\bibitem{ICB5}
Baianu, I.C.: 1973, Some Algebraic Properties of \emph{\textbf{(M,R)}} -- Systems. \emph{Bulletin of Mathematical Biophysics} \textbf{35}, 213-217.

\bibitem{ICBm2}
Baianu, I.C. and M. Marinescu: 1974, On A Functorial Construction of \emph{\textbf{(M,R)}}-- Systems. \emph{Revue Roumaine de Mathematiques Pures et Appliquees} \textbf{19}: 388-391.

\bibitem{ICB6}
Baianu, I.C.: 1977, A Logical Model of Genetic Activities in \L ukasiewicz Algebras: The Non-linear Theory. \emph{Bulletin of Mathematical Biology},
\textbf{39}: 249-258.

\bibitem{ICB2}
Baianu, I.C.: 1980a, Natural Transformations of Organismic Structures.,
\emph{Bulletin of Mathematical Biology},\textbf{42}: 431-446.

\bibitem{Bgg2}
Baianu, I. C., Glazebrook, J. F. and G. Georgescu: 2004, Categories of Quantum Automata and N-Valued \L ukasiewicz Algebras in Relation to Dynamic Bionetworks, \textbf{(M,R)}--Systems and Their Higher Dimensional Algebra, \emph{Abstract and Preprint of Report}: $\\http://www.ag.uiuc.edu/fs401/QAuto.pdf $ and $http://www.medicalupapers.com/quantum+automata+math+categories+baianu/$

\bibitem{BBGG1}
Baianu I. C., Brown R., Georgescu G. and J. F. Glazebrook: 2006, Complex Nonlinear Biodynamics in Categories, Higher Dimensional Algebra and \L ukasiewicz-Moisil Topos: Transformations of Neuronal, Genetic and Neoplastic Networks., \emph{Axiomathes}, \textbf{16} Nos. 1-2: 65-122.

\bibitem{Bbg3}
Baianu, I.C., R. Brown and J.F. Glazebrook. : 2007a, Categorical Ontology of Complex Spacetime Structures: The Emergence of Life and Human Consciousness, Axiomathes, 17: 35-168.

\bibitem{Bggb4}
Baianu, I.C.,  R. Brown and J. F. Glazebrook: 2007b, A Non-Abelian, Categorical Ontology of Spacetimes and Quantum Gravity, Axiomathes, 17: 169-225.

\bibitem{Ba-We85}
Barr, M. and Wells, C., 1985, Toposes, Triples and Theories, New York: Springer-Verlag.
 
\bibitem{BM-CW99}
Barr, M. and Wells, C., 1999, Category Theory for Computing Science, Montreal: CRM. 

\bibitem{BJL81}
Bell, J. L., 1981, ``Category Theory and the Foundations of Mathematics'', British Journal for the Philosophy of Science, 32, 349--358. 
 
\bibitem{BJL82}
Bell, J. L., 1982, ``Categories, Toposes and Sets'', Synthese, 51, 3, 293--337. 
 
\bibitem{BJL86}
Bell, J. L., 1986, ``From Absolute to Local Mathematics'', Synthese, 69, 3, 409--426. 

\bibitem{BJL88} 
Bell, J. L., 1988, Toposes and Local Set Theories: An Introduction, Oxford: Oxford University Press. 

\bibitem{BG-MCLS99}
Birkoff, G. \& Mac Lane, S., 1999, Algebra, 3rd ed., Providence: AMS.  

\bibitem{BA-SA83}
Blass, A. and Scedrov, A., 1983, Classifying Topoi and Finite Forcing , Journal of Pure and Applied Algebra, 28, 111--140. 

\bibitem{BASA92}
Blass, A. and Scedrov, A., 1992, "Complete Topoi Representing Models of Set Theory", Annals of Pure and Applied Logic , 57, no. 1, 1--26.  


\bibitem{Borceux94}
Borceux, F.: 1994, \emph{Handbook of Categorical Algebra}, vols: 1--3, 
in {\em Encyclopedia of Mathematics and its Applications} \textbf{50} to \textbf{52}, Cambridge University Press.

\bibitem{Bourbaki1}
Bourbaki, N. 1961 and 1964: \emph{Alg\`{e}bre commutative.},
in \`{E}l\'{e}ments de Math\'{e}matique., Chs. 1--6., Hermann: Paris.

\bibitem (BJk4)
Brown, R. and G. Janelidze: 2004, Galois theory and a new homotopy
double groupoid of a map of spaces, \emph{Applied Categorical
Structures} \textbf{12}: 63-80.

\bibitem{BHR2}
Brown, R., Higgins, P. J. and R. Sivera,: 2007a, \emph{Non-Abelian
Algebraic Topology}, in preparation.\\
http://www.bangor.ac.uk/~mas010/nonab-a-t.html ; \\
http://www.bangor.ac.uk/~mas010/nonab-t/partI010604.pdf

\bibitem{BGB2k7b}
Brown, R., Glazebrook, J. F. and I.C. Baianu.: 2007b, A Conceptual, Categorical and Higher Dimensional Algebra Framework of Universal Ontology and the Theory of Levels for Highly Complex Structures and Dynamics., \emph{Axiomathes} (17): 321--379.

\bibitem{BP2k3}
Brown R. and T. Porter: 2003, Category theory and higher
dimensional algebra: potential descriptive tools in neuroscience, In:
Proceedings of the International Conference on Theoretical
Neurobiology, Delhi, February 2003, edited by Nandini Singh,
National Brain Research Centre, Conference Proceedings 1, 80-92.

\bibitem{Br-Har-Ka-Po2k2}
Brown, R., Hardie, K., Kamps, H. and T. Porter: 2002, The homotopy
double groupoid of a Hausdorff space., \emph{Theory and
Applications of Categories} \textbf{10}, 71-93.

\bibitem{Br-Hardy76}
Brown, R., and Hardy, J.P.L.:1976, Topological groupoids I:
universal constructions, \emph{Math. Nachr.}, 71: 273-286.

\bibitem{Br-Sp76}
Brown, R. and Spencer, C.B.: 1976, Double groupoids and crossed
modules, \emph{Cah.  Top. G\'{e}om. Diff.} \textbf{17}, 343-362.

\bibitem{BR-SCB76}
Brown R, Razak Salleh A (1999) Free crossed resolutions of groups and presentations of modules of
identities among relations. {\em LMS J. Comput. Math.}, \textbf{2}: 25--61.

\bibitem{BDA55}
Buchsbaum, D. A.: 1955, Exact categories and duality., Trans. Amer. Math. Soc. \textbf{80}: 1-34.

\bibitem{BDA55}
Buchsbaum, D. A.: 1969, A note on homology in categories., Ann. of Math. \textbf{69}: 66-74.

\bibitem{BI-DA68}
Bucur, I., and Deleanu A. (1968). {\em  Introduction to the Theory of Categories and Functors}. J.Wiley and Sons: London

\bibitem{BL2k3}
Bunge, M. and S. Lack: 2003, Van Kampen theorems for toposes, \emph{Adv. in Math.} \textbf{179}, 291-317.


\bibitem{BM84}
Bunge, M., 1984, "Toposes in Logic and Logic in Toposes", Topoi, 3, no. 1, 13-22. 

\bibitem{BM-LS2k3}
Bunge M, Lack S (2003) Van Kampen theorems for toposes. {\em Adv Math}, \textbf {179}: 291-317.

\bibitem{CH-ES56}
Cartan, H. and Eilenberg, S. 1956. {\em Homological Algebra}, Princeton Univ. Press: Pinceton. 


\bibitem{CPM65}
Cohen, P.M. 1965. {\em Universal Algebra}, Harper and Row: New York, London and Tokyo.

\bibitem{CA94}
Connes A 1994. \emph{Noncommutative geometry}. Academic Press: New York.

\bibitem{CR-LL63}
Croisot, R. and Lesieur, L. 1963. \emph{Alg\`ebre noeth\'erienne non-commutative.},
Gauthier-Villard: Paris.

\end{thebibliography}

\textbf{...more to come}

%%%%%
%%%%%
\end{document}
