\documentclass[12pt]{article}
\usepackage{pmmeta}
\pmcanonicalname{Quandles}
\pmcreated{2013-03-22 16:42:37}
\pmmodified{2013-03-22 16:42:37}
\pmowner{StevieHair}{1420}
\pmmodifier{StevieHair}{1420}
\pmtitle{quandles}
\pmrecord{8}{38926}
\pmprivacy{1}
\pmauthor{StevieHair}{1420}
\pmtype{Definition}
\pmcomment{trigger rebuild}
\pmclassification{msc}{08A99}

\endmetadata

% this is the default PlanetMath preamble.  as your knowledge
% of TeX increases, you will probably want to edit this, but
% it should be fine as is for beginners.

% almost certainly you want these

\usepackage{amssymb}
\usepackage{amsmath}
\usepackage{amsfonts}

% used for TeXing text within eps files
%\usepackage{psfrag}
% need this for including graphics (\includegraphics)
%\usepackage{graphicx}
% for neatly defining theorems and propositions
%\usepackage{amsthm}
% making logically defined graphics
%%%\usepackage{xypic}

% there are many more packages, add them here as you need them

% define commands here

\newtheorem{defn}{Definition}

\begin{document}
Quandles are algebraic gadgets introduced by David Joyce in \cite{Joyce} which can be used to define invarients of links. In the case of knots these invarients are complete up to equivalence, that is up to mirror images.
\begin{defn}
A \emph{quandle} is an algebraic structure, specifically it is a set $Q$ with two binary operations on it, $\lhd$ and $\lhd^{-1}$ and the following axioms.
\begin{enumerate}
  \item $q \lhd q = q \ \forall q \in Q$
  \item $(q_1 \lhd q_2) \lhd^{-1} q_2 = (q_1 \lhd^{-1} q_2)\lhd q_2 \ \forall q_1,q_2 \in Q$
  \item $(q_1 \lhd q_2) \lhd q_3 = (q_1 \lhd q_3) \lhd (q_2 \lhd q_3) \ \forall q_1,q_2,q_3 \in Q$
\end{enumerate}
\end{defn}
It is useful to consider $q_1 \lhd q_2$ as '$q_2$ acting on $q_1$'. \newline
Examples. \newline
\begin{enumerate}
  \item Let $Q$ be some group, and let $n$ be some fixed integer. 
Then let $g_1 \lhd g_2 = g_2^{-n}g_1g_2^n, \hspace{8pt} g_1 \lhd^{-1} g_2 = g_2^ng_1g_2^{-n}$.
  \item Let $Q$ be some group. Then let $g_1 \lhd g_2 = g_1 \lhd^{-1}g_2 = g_2g_1^{-1}g_2$.
  \item Let $Q$ be some module, and $T$ some invertable linear operator on $Q$. Then let 
      $m_1 \lhd m_2 = T(m_1 - m_2) +m_2, \hspace{8pt} m_1 \lhd^{-1} m_2 = T^{-1}(m_1 -m_2)+m_2$ 
\end{enumerate}
Homomorphisms, isomorphisms etc. are defined in the obvious way.
Notice that the third axiom gives us that the operation of a quandle element on the quandle given by
$f_q\colon q' \mapsto q' \lhd q$ is a homomorphism, and the second axiom ensures that this is an isomorphism. 
\begin{defn}
The subgroup of the automorphism group of a quandle $Q$ generated by the quandle operations is the \emph{operator group} of $Q$.  
\end{defn}

\begin{thebibliography}{99}
\bibitem{Joyce} D.Joyce : A Classifying Invariant Of Knots, The Knot 
Quandle : J.P.App.Alg 23 (1982) 37-65 
\end{thebibliography}

%%%%%
%%%%%
\end{document}
