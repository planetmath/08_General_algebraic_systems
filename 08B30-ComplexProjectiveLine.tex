\documentclass[12pt]{article}
\usepackage{pmmeta}
\pmcanonicalname{ComplexProjectiveLine}
\pmcreated{2013-03-22 13:59:54}
\pmmodified{2013-03-22 13:59:54}
\pmowner{bwebste}{988}
\pmmodifier{bwebste}{988}
\pmtitle{complex projective line}
\pmrecord{7}{34818}
\pmprivacy{1}
\pmauthor{bwebste}{988}
\pmtype{Definition}
\pmcomment{trigger rebuild}
\pmclassification{msc}{08B30}

\usepackage{amssymb}
\usepackage{amsmath}
\usepackage{amsfonts}
\begin{document}
Let $\mathbb C$ be the set of complex numbers. We define an equivalence relation on $\mathbb C^2-\{0,0\}$ by 
\begin{equation}
(x1,y1) \sim (x2,y2) \Leftrightarrow \exists \lambda \in \mathbb C^* \lambda (x1,y1)=(x2,y2)
\end{equation}

The set $\mathbb C^2-\{0,0\}/\sim$ is a projective variety called the complex projective line.
%%%%%
%%%%%
\end{document}
