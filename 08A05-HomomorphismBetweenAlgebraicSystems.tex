\documentclass[12pt]{article}
\usepackage{pmmeta}
\pmcanonicalname{HomomorphismBetweenAlgebraicSystems}
\pmcreated{2013-03-22 15:55:36}
\pmmodified{2013-03-22 15:55:36}
\pmowner{CWoo}{3771}
\pmmodifier{CWoo}{3771}
\pmtitle{homomorphism between algebraic systems}
\pmrecord{8}{37934}
\pmprivacy{1}
\pmauthor{CWoo}{3771}
\pmtype{Definition}
\pmcomment{trigger rebuild}
\pmclassification{msc}{08A05}
\pmdefines{compatible function}
\pmdefines{homomorphism}
\pmdefines{monomorphism}
\pmdefines{epimorphism}
\pmdefines{endomorphism}
\pmdefines{isomorphism}
\pmdefines{automorphism}
\pmdefines{homomorphic image}

\usepackage{amssymb,amscd}
\usepackage{amsmath}
\usepackage{amsfonts}

% used for TeXing text within eps files
%\usepackage{psfrag}
% need this for including graphics (\includegraphics)
%\usepackage{graphicx}
% for neatly defining theorems and propositions
%\usepackage{amsthm}
% making logically defined graphics
%%%\usepackage{xypic}

% define commands here

\begin{document}
Let $(A,O),(B,O)$ be two algebraic systems with operator set $O$.  Given operators $\omega_A$ on $A$ and $\omega_B$ on $B$, with $\omega\in O$ and $n=$ arity of $\omega$, a function $f:A\to B$ is said to be \emph{compatible} with $\omega$ if $$f(\omega_A(a_1,\ldots,a_n))=\omega_B(f(a_1),\ldots,f(a_n)).$$

Dropping the subscript, we now simply identify $\omega\in O$ as an operator for both algebras $A$ and $B$.  If a function $f:A\to B$ is compatible with every operator $\omega\in O$, then we say that $f$ is a \emph{homomorphism} from $A$ to $B$.  If $O$ contains a constant operator $\omega$ such that $a\in A$ and $b\in B$ are two constants assigned by $\omega$, then any homomorphism $f$ from $A$ to $B$ maps $a$ to $b$.

\textbf{Examples}.
\begin{enumerate}
\item When $O$ is the empty set, any function from $A$ to $B$ is a homomorphism.  
\item When $O$ is a singleton consisting of a constant operator, a homomorphism is then a function $f$ from one pointed set $(A,p)$ to another $(B,q)$, such that $f(p)=q$.
\item A homomorphism defined in any one of the well known algebraic systems, such as groups, modules, rings, and \PMlinkname{lattices}{Lattice} is consistent with the more general definition given here.  The essential thing to remember is that a homomorphism preserves constants, so that between two rings with 1, both the additive identity 0 and the multiplicative identity 1 are preserved by this homomorphism.  Similarly, a homomorphism between two \PMlinkname{bounded lattices}{BoundedLattice} is called a $\lbrace 0,1\rbrace$-\PMlinkname{lattice homomorphism}{LatticeHomomorphism} because it preserves both 0 and 1, the bottom and top elements of the lattices.
\end{enumerate}

\textbf{Remarks}.  
\begin{itemize}
\item
Like the familiar algebras, once a homomorphism is defined, special types of homomorphisms can now be named: 
\begin{itemize}
\item a homomorphism that is one-to-one is a \emph{monomorphism}; 
\item an onto homomorphism is an \emph{epimorphism}; 
\item an \emph{isomorphism} is both a monomorphism and an epimorphism; 
\item a homomorphism such that its codomain is its domain is called an \emph{endomorphism}; 
\item finally, an \emph{automorphism} is an endomorphism that is also an isomorphism.
\end{itemize}
\item All trivial algebraic systems (of the same type) are isomorphic.
\item
If $f:A\to B$ is a homomorphism, then the image $f(A)$ is a subalgebra of $B$.  If $\omega_B$ is an $n$-ary operator on $B$, and $c_1,\ldots,c_n\in f(A)$, then $\omega_B(c_1,\ldots,c_n)=\omega_B(f(a_1),\ldots,f(a_n))=f(\omega_A(a_1,\ldots, a_n))\in f(A)$.  $f(A)$ is sometimes called the \emph{homomorphic image} of $f$ in $B$ to emphasize the fact that $f$ is a homomorphism.
\end{itemize}
%%%%%
%%%%%
\end{document}
