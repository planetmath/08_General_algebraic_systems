\documentclass[12pt]{article}
\usepackage{pmmeta}
\pmcanonicalname{ClosureOfASubsetUnderRelations}
\pmcreated{2013-03-22 16:21:26}
\pmmodified{2013-03-22 16:21:26}
\pmowner{CWoo}{3771}
\pmmodifier{CWoo}{3771}
\pmtitle{closure of a subset under relations}
\pmrecord{36}{38492}
\pmprivacy{1}
\pmauthor{CWoo}{3771}
\pmtype{Definition}
\pmcomment{trigger rebuild}
\pmclassification{msc}{08A02}
\pmdefines{closed under}
\pmdefines{closure property}

\usepackage{amssymb,amscd}
\usepackage{amsmath}
\usepackage{amsfonts}

% used for TeXing text within eps files
%\usepackage{psfrag}
% need this for including graphics (\includegraphics)
%\usepackage{graphicx}
% for neatly defining theorems and propositions
\usepackage{amsthm}
% making logically defined graphics
%%\usepackage{xypic}
\usepackage{pst-plot}
\usepackage{psfrag}

% define commands here
\newtheorem{prop}{Proposition}
\newtheorem{thm}{Theorem}
\newtheorem{ex}{Example}
\newcommand{\real}{\mathbb{R}}
\newcommand{\pdiff}[2]{\frac{\partial #1}{\partial #2}}
\newcommand{\mpdiff}[3]{\frac{\partial^#1 #2}{\partial #3^#1}}
\begin{document}
Let $A$ be a set and $R$ be an $n$-ary relation on $A$, $n\ge 1$.  A subset $B$ of $A$ is said to be \emph{closed under $R$}, or \emph{$R$-closed}, if, whenever $b_1,\ldots,b_{n-1}\in B$, and $(b_1,\ldots,b_{n-1},b_n)\in R$, then $b_n\in B$.  

Note that if $R$ is unary, then $B$ is $R$-closed iff $R\subseteq B$.

More generally, let $A$ be a set and $\mathcal{R}$ a set of (finitary) relations on $A$.  A subset $B$ is said to be \emph{$\mathcal{R}$-closed} if $B$ is $R$-closed for each $R\in\mathcal{R}$.

Given $B\subseteq A$ with a set of relations $\mathcal{R}$ on $A$, we say that $C\subseteq A$ is an \emph{$\mathcal{R}$-closure} of $B$ if 
\begin{enumerate}
\item $B\subseteq C$,
\item $C$ is $\mathcal{R}$-closed, and 
\item if $D\subseteq A$ satisfies both 1 and 2, then $C\subseteq D$.
\end{enumerate}

By condition 3, $C$, if exists, must be unique.  Let us call it the $\mathcal{R}$-closure of $B$, and denote it by $\operatorname{Cl}_{\mathcal{R}}(B)$.  If $\mathcal{R}=\lbrace R\rbrace$, then we call it the $R$-closure of $B$, and denote it by $\operatorname{Cl}_R(B)$ correspondingly.

Here are some examples.
\begin{enumerate}
\item Let $A=\mathbb{Z}$, $B=\lbrace 5\rbrace$, and $R$ be the relation that $mRn$ whenever $m$ divides $n$.  Clearly $B$ is not closed under $R$ (for example, $(5,10)\in R$ but $10\notin B$).  Then $\operatorname{Cl}_{R}(B) = 5\mathbb{Z}$.  If $R$ is instead the relation $\le$, then $\operatorname{Cl}_{\le}(B)=\lbrace n\in A\mid n\ge 5\rbrace$.
\item
This is an example where $R$ is in fact a function (operation).  Suppose $A=\mathbb{Z}$ and $R$ is the binary operation subtraction $-$.  Suppose $B=\lbrace 3,5\rbrace$.  Then $\operatorname{Cl}_{-}(B)=A$.  To see this, set $C=\operatorname{Cl}_{-}(B)$.  Note that $2=5-3\in C$ so $1=3-2$ as well as $-1=2-3\in C$.  This means that if $n\in C$, both $n-1$ and $n+1\in C$.  By induction, $C=\mathbb{Z}$.  In general, if $B=\lbrace p,q\rbrace$, where $p,q$ are coprime, then $\operatorname{Cl}_{-} (B) =\mathbb{Z}$.  This is essentially the result of the Chinese Remainder Theorem.
\item
If $R$ is unary, then the $R$-closure of $B\subseteq A$ is just $B\cup R$.  When every $R\in \mathcal{R}$ is unary, then the $\mathcal{R}$-closure of $B$ in $A$ is $(\bigcup \mathcal{R}) \cup B$.
\end{enumerate}

\begin{prop} $\operatorname{Cl}_{\mathcal{R}}(B)$ exists for every $B\subseteq A$. \end{prop}
\begin{proof}
Let $S_{\mathcal{R}}(B)$ be the set of subsets of $A$ satisfying the defining conditions 1 and 2 of $\mathcal{R}$-closures above, partially ordered by $\subseteq$.  If $\mathcal{C} \subseteq S_{\mathcal{R}}(B)$, then $\bigcap \mathcal{C} \in S_{\mathcal{R}}(B)$.  To see this, we break the statement down into cases:
\begin{itemize}
\item 
In the case when $\mathcal{C}=\varnothing$, we have $\bigcap \mathcal{C} = A\in S_{\mathcal{R}}(B)$.  
\item
When $C:=\bigcap \mathcal{C}\ne \varnothing$, pick any $n$-ary relation $R\in \mathcal{R}$.  
\begin{enumerate}
\item
If $n=1$, then, since aach $D\in \mathcal{C}$ is $R$-closed, $R\subseteq D$.  Therefore, $R\subseteq \bigcap \mathcal{C} = \bigcap \lbrace D \mid D \in \mathcal{C}\rbrace = C$.  So $C$ is $R$-closed.
\item
If $n>1$, pick elements $c_1,\ldots, c_{n-1}\in C$ such that $(c_1,\ldots, c_n)\in R$.  As each $c_i\in D$ for $i=1,\ldots, n-1$, and $D$ is $R$-closed, $c_n\in D$.  Since $c_n\in D$ for every $D\in \mathcal{C}$, $c_n\in C$ as well.  This shows that $C$ is $R$-closed.  
\end{enumerate}
In both cases, $B\subseteq C$ since $B\subseteq D$ for every $D\in \mathcal{C}$.  Therefore, $C\in S_{\mathcal{R}}(B)$.
\end{itemize}
Hence, $S_{\mathcal{R}}(B)$ is a complete lattice by virtue of \PMlinkname{this fact}{CriteriaForAPosetToBeACompleteLattice}, which means that $S_{\mathcal{R}}(B)$ has a minimal element, which is none other than the $\mathcal{R}$-closure $\operatorname{Cl}_{\mathcal{R}}(B)$ of $B$.
\end{proof}

\textbf{Remark}.  It is not hard to see $\operatorname{Cl}_{\mathcal{R}}$ has the following properties:
\begin{enumerate}
\item $B\subseteq \operatorname{Cl}_{\mathcal{R}}(B)$,
\item $\operatorname{Cl}_{\mathcal{R}}(\operatorname{Cl}_{\mathcal{R}}(B))= \operatorname{Cl}_{\mathcal{R}}(B)$, and
\item if $B\subseteq C$, then $\operatorname{Cl}_{\mathcal{R}}(B)\subseteq \operatorname{Cl}_{\mathcal{R}}(C)$.
\end{enumerate}

Next, assume that $\mathcal{S}$ is another set of finitary relations on $A$.  Then
\begin{enumerate}
\item if $\mathcal{R}\subseteq \mathcal{S}$, then $\operatorname{Cl}_{\mathcal{S}}(B) \subseteq \operatorname{Cl}_{\mathcal{R}}(B)$,
\item $\operatorname{Cl}_{\mathcal{S}}(\operatorname{Cl}_{\mathcal{R}}(B)) \subseteq  \operatorname{Cl}_{\mathcal{R}\cap\mathcal{S}}(B)$, and 
\item $\operatorname{Cl}_{\mathcal{S}}(\operatorname{Cl}_{\mathcal{R}}(B))= \operatorname{Cl}_{\mathcal{R}\cap\mathcal{S}}(B)$ if $\mathcal{R}\subseteq \mathcal{S}$ or $\mathcal{S}\subseteq \mathcal{R}$.
\end{enumerate}
%%%%%
%%%%%
\end{document}
