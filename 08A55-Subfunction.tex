\documentclass[12pt]{article}
\usepackage{pmmeta}
\pmcanonicalname{Subfunction}
\pmcreated{2013-03-22 18:41:54}
\pmmodified{2013-03-22 18:41:54}
\pmowner{CWoo}{3771}
\pmmodifier{CWoo}{3771}
\pmtitle{subfunction}
\pmrecord{8}{41459}
\pmprivacy{1}
\pmauthor{CWoo}{3771}
\pmtype{Definition}
\pmcomment{trigger rebuild}
\pmclassification{msc}{08A55}
\pmclassification{msc}{03E20}
\pmdefines{restriction}

\usepackage{amssymb,amscd}
\usepackage{amsmath}
\usepackage{amsfonts}
\usepackage{mathrsfs}

% used for TeXing text within eps files
%\usepackage{psfrag}
% need this for including graphics (\includegraphics)
%\usepackage{graphicx}
% for neatly defining theorems and propositions
\usepackage{amsthm}
% making logically defined graphics
%%\usepackage{xypic}
\usepackage{pst-plot}

% define commands here
\newcommand*{\abs}[1]{\left\lvert #1\right\rvert}
\newtheorem{prop}{Proposition}
\newtheorem{thm}{Theorem}
\newtheorem{ex}{Example}
\newcommand{\real}{\mathbb{R}}
\newcommand{\pdiff}[2]{\frac{\partial #1}{\partial #2}}
\newcommand{\mpdiff}[3]{\frac{\partial^#1 #2}{\partial #3^#1}}
\begin{document}
\textbf{Definition}.  Let $f:A\to B$ and $g:C\to D$ be partial functions.  $g$ is said to be a \emph{subfunction} of $f$ if $$g\subseteq f \cap (C\times D).$$

In other words, $g$ is a subfunction of $f$ iff whenever $x\in C$ such that $g(x)$ is defined, then $x\in A$, $f(x)$ is defined, and $g(x)=f(x)$.

If we set $C'=A\cap C$ and $D'=B\cap D$, then $g\subseteq f\cap (C'\times D')$, so there is no harm in assuming that $C$ and $D$ are subsets of $A$ and $B$ respectively, which we will do for the rest of the discussion.

In practice, whenever $g$ is a subfunction of $f$, we often assume that $g$ and $f$ have the same domain and codomain.  Otherwise, we would specify that $g$ is a subfunction of $f:A\to B$ with domain $C$ and codomain $D$.

For example, $f:\mathbb{R} \to \mathbb{R}$ defined by $$f(x)=\sqrt{x^2-1}$$ is a partial function, whose domain of definition is $(-\infty,-1]\cup [1,\infty)$, and the partial function $g:\mathbb{R} \to \mathbb{R}$ given by 
$$g(x)=\displaystyle{\frac{x^2-1}{\sqrt{x^2-1}}}$$ is a subfunction of $f$.  The domain of definition of $g$ is $(-\infty,-1)\cup (1,\infty)$.

Two immediate properties of a subfunction $g:C\to D$ of $f:A\to B$ are 
\begin{itemize}
\item
the range of $g$ is a subset of the range of $f$: $$g(C)\subseteq f(C),$$
\item
the domain of definition of $g$ is a subset of the domain of definition of $f$: $$g^{-1}(D)\subseteq f^{-1}(D).$$
\end{itemize}

\textbf{Definition}.  A subfunction $g:C\to D$ of $f:A\to B$ is called a \emph{restriction of $f$ relative to $D$}, if $g(C)=f(C)\cap D$, and a \emph{restriction of $f$} if $g(C)=f(C)$.

Every partial function $g:C\to D$ corresponds to a unique restriction $g':C\to g(C)$ of $g$.

A restriction $g:C\to D$ of $f:A\to B$ is certainly a restriction of $f$ relative to $D$, since $f(C)\cap D = g(C)\cap D = g(C)$, but not conversely.  For example, let $A$ be the set of all non-negative integers and $-_A: A^2\to A$ the ordinary subtraction.  $-_A$ is easily seen to be a partial function.  Let $B$ be the set of all positive integers.  Then $-_B:B^2\to B$ is a restriction of $-_A:A^2\to A$, relative to $B$.  However, $-_B$ is not a restriction of $-_A$, for $n -_B n$ is not defined, while $n -_A n = 0\in A$.

\begin{thebibliography}{7}
\bibitem{gg} G. Gr\"{a}tzer: {\em Universal Algebra}, 2nd Edition, Springer, New York (1978).
\end{thebibliography}
%%%%%
%%%%%
\end{document}
