\documentclass[12pt]{article}
\usepackage{pmmeta}
\pmcanonicalname{Variation}
\pmcreated{2013-03-22 14:53:49}
\pmmodified{2013-03-22 14:53:49}
\pmowner{drini}{3}
\pmmodifier{drini}{3}
\pmtitle{variation}
\pmrecord{7}{36579}
\pmprivacy{1}
\pmauthor{drini}{3}
\pmtype{Topic}
\pmcomment{trigger rebuild}
\pmclassification{msc}{08C99}
\pmsynonym{Proportion}{Variation}
\pmrelated{HomogeneousEquation}
\pmrelated{GraphOfEquationXyConstant}
\pmrelated{ProportionalityOfNumbers}
\pmdefines{Relationships between two or more variables.}

\endmetadata

% this is the default PlanetMath preamble.  as your knowledge
% of TeX increases, you will probably want to edit this, but
% it should be fine as is for beginners.

% almost certainly you want these
\usepackage{amssymb}
\usepackage{amsmath}
\usepackage{amsfonts}

% used for TeXing text within eps files
%\usepackage{psfrag}
% need this for including graphics (\includegraphics)
%\usepackage{graphicx}
% for neatly defining theorems and propositions
%\usepackage{amsthm}
% making logically defined graphics
%%%\usepackage{xypic}

% there are many more packages, add them here as you need them

% define commands here
\begin{document}
Variation and proportion are defined to be the relationship between two or more variables with regard to a constant of proportionality.

The traditional notation for direct proportionality is $x \propto y$ or, if using regular equality notation, $x = ky$.

Here, $k$ denotes the constant of proportionality.

Similarly, the traditional notation for inverse proportionality is $x \propto 1/y$ or, with regular equality, $x = k/y$.

For direct proportionality, to find the value of an unknown $x$ or $y$, you may use the formula:
$y_{1}/x_{1} = y_{2}/x_{2}$

Similarly, for inverse proportion it would be:
$x_{1}/y_{1} = y_{2}/x_{2}$
%%%%%
%%%%%
\end{document}
